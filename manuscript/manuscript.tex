% !TeX program = pdfLaTeX
\documentclass[smallextended]{svjour3}       % onecolumn (second format)
%\documentclass[twocolumn]{svjour3}          % twocolumn
%
\smartqed  % flush right qed marks, e.g. at end of proof
%
\usepackage{amsmath}
\usepackage{graphicx}
\usepackage[utf8]{inputenc}

\usepackage[hyphens]{url} % not crucial - just used below for the URL
\usepackage{hyperref}
\providecommand{\tightlist}{%
  \setlength{\itemsep}{0pt}\setlength{\parskip}{0pt}}

%
% \usepackage{mathptmx}      % use Times fonts if available on your TeX system
%
% insert here the call for the packages your document requires
%\usepackage{latexsym}
% etc.
%
% please place your own definitions here and don't use \def but
% \newcommand{}{}
%
% Insert the name of "your journal" with
% \journalname{myjournal}
%

%% load any required packages here




\usepackage{float}
\floatplacement{figure}{tb}

\begin{document}

\title{The dynamics of risk perception in a Mediterranean agroecosystem \thanks{Grants or other notes about the article that should go on the front page
should be placed here. General acknowledgments should be placed at the
end of the article.} }


    \titlerunning{The dynamics of risk perception in a Mediterranean agroecosystem}

\author{  Nicolas Gauthier 1 \and  }

    \authorrunning{ Gauthier }

\institute{
        Nicolas Gauthier 1 \at
     Laboratory of Tree-Ring Research \& School of Geography and Development,
 University of Arizona \\
     \email{\href{mailto:ngauthier@email.arizona.edu}{\nolinkurl{ngauthier@email.arizona.edu}}}  %  \\
%             \emph{Present address:} of F. Author  %  if needed
    \and
    }

\date{Received: date / Accepted: date}
% The correct dates will be entered by the editor


\maketitle

\begin{abstract}
Small-scale agriculturalists in the Mediterranean Basin rely on multiple
strategies including diversification, intensification, and storage to
maintain a stable food supply in the face of environmental uncertainty.
Each of these strategies requires farmers to make specific resource
allocation decisions in response to environmental risks and is thus
sensitive to variability in both the spatiotemporal pattern of risk and
the ability of farmers to perceive that pattern. In this chapter, I
present an agent-based model of a Mediterranean agroecosystem. By
driving the model with realistic environmental dynamics derived from
simulations of mid-Holocene Mediterranean climate, and by allowing the
psychology of risk perception to vary among individual farmers, I
explore the hidden vulnerabilities of traditional risk-management
strategies to periods of rapid climate change. I show that even when
farmers are able to manage risk ``optimally'' in light of past
experience, unanticipated changes in the spatiotemporal pattern of
rainfall can still lead to major food shortfalls.
\\
\keywords{
        key \and
        dictionary \and
        word \and
    }

    \subclass{
                    MSC code 1 \and
                    MSC code 2 \and
            }

\end{abstract}


\def\spacingset#1{\renewcommand{\baselinestretch}%
{#1}\small\normalsize} \spacingset{1}


\hypertarget{intro}{%
\section{Introduction}\label{intro}}

Over the past 10,000 years, subsistence farmers in the Mediterranean
basin have developed sophisticated strategies to maintain stable food
supplies given uncertain rainfall. Their suite of strategies include
practices like crop diversification, storage, and exchange (Halstead and
O'Shea 1989)). These risk-management strategies all represent forms of
\emph{regulatory feedback} used to manage complex socio-ecological
systems. Regulatory feedbacks convert a system's high-variance inputs
into low-variance outputs by dynamically monitoring the outputs and
adjusting internal properties of the system ({\textbf{???}}).
Risk-managing strategies require farmers to make specific resource
allocation decisions in response to environmental risks, and as
regulatory feedbacks they are thus sensitive to variability in both the
spatiotemporal patterns of risk and the ability of farmers to perceive
and act on those patterns.

Crop diversification is an excellent example of a widespread and
effective regulatory feedback strategy (Figure 1). Relying on a mix of
food types with different climatic tolerances is an efficient way to
maintain a robust food supply ({\textbf{???}},Anderies2006). In the
Mediterranean, land-use strategies involving a diversified portfolio of
wheat and barley have been employed by even the earliest sedentary
farmers, and continue to be used to this day (Gould
1963,Slafer1999,Abbo2009a,Marston2011190). Wheat generally has higher
yields but is sensitive to water deficits, while barley has lower yields
and is drought tolerant. Planting a mix of wheat and barley, either in
the same plot or in a combination of plots, is an effective means of
diversifying grain supplies. By adjusting the ratio of wheat to barley,
farmers can adapt to changing drought risks.

The secret to the success of earliest farmers on the mediterrenan
littoral was in their manipulation of the local ecosystem to manage
environmental risks. Early farmers constructed a complex agroecosystem
to manage risk. They needed to maintain stable crop yields in face of
uncertain drought risks. Yield stability is very important for
traditional agriculturlal commmunities not connected to larger bulk food
networks (Abbo et al 2010) neolithic crop package -- weiss zohary 2011
cite abbo et al 2010? for more stuff about ladraces Wheat/barley systems
example of system that combines low yield, low wirks crop with high
yield high risk crop, an effective context paut et al 2019 These
strategies work when risks are knonw and constant, but may be more
susceptible to periods of changing climateic risks so understanding how
people of the past adapted to change is important to us today

Are these crop-diversification strategies vulnerable to the same
dynamics as other social-ecological systems with similar regulatory
feedbacks mechanisms? One way to address this is to model the influence
of imperfect monitoring and biased decision making in uncertain
environments. In this study, I accomplish this by answering two main
questions:

\begin{enumerate}
\def\labelenumi{\arabic{enumi}.}
\tightlist
\item
  How likely are droughts to occur each year in the eastern
  Mediterreanean, and how did these risks change over the Holocene?
\item
  How might Neolithic farmers have percieved these changing drought
  risks, and what were the consequences for farming communities with
  respect to their collective ability to manage risk?
\end{enumerate}

I couple a climate model with high temporal resolution to a simple game
theoretic model of agricultural decision making under uncertainty, in
order to examine the performance of optimal and suboptimal wheat-barley
diversification practices. I model the year-to-year crop diversification
strategies made by farmers as an iterated game of fictitious play
against nature (Gould 1963). Farmers are boundedly rational, in that
they seek to maximize their objective functions given the perceived
probabilities of different actions by nature, but work with flawed
mental models of climate-related risks due to imperfect recall of past
events. Furthermore, by using a climate model to represent the actions
of nature, I can more precisely capture the characteristic
autocorrelated patterns of rainfall variability, rather than simply
drawing rainfall values from a static distribution. This framework
allows me to address the sensitivity of diversification strategies to
changes and climatic variability and imperfect human perceptions of
those changes.

\hypertarget{sec:1}{%
\section{Decision-making in a game against nature}\label{sec:1}}

The basic decision-making problem facing a farmer seeking to diversify
their crops can be thought of as a game against nature. We can represent
this simplified decision context as a ``game'' in a game theorietic
context (Luce raiffa 1989, milnor 1952, Agrawal1968, cassidy1971, gould
1963). The ``game'' in this context is the farmer's decision of which
crops to plant and in what proportions, given uncertianty in the future
``state-of-nature'', and nature varies between several possible states
such as dry and wet years (Table 1). In reality continuous, but people
often intuitively solve an easier problem when faced with a complex
real-world situation (cite behavioral econ or simon). The exact values
in the payoff matrix here are less important than the relative payoffs
in each quadrant. We can think of this game as being the
culturally-inherited object. Our question is how do individuals make
decisions here? So in this case the cultural norm would be to mix wheat
and barley, but the learning is in the exact ratio.

\begin{table}
\centering
\caption{Estimates of yield volume (t/ha) for ancient wheat and barley varieties derived from [@Slafer1999].}
\begin{tabular}{|l|l|l|}
\hline

 & Dry Year & Normal Year \\ \hline
Barley Yield & 0.93 & 1.18 \\ \hline
Wheat Yield & 0 & 1.60 \\ \hline

\end{tabular}
\end{table}

How should an early farmer make a decision in light ot this uncertainty?
The basic logic of crop diversitication is Portfolio of crops ala modern
portfolio theory: Blank2001 Sometimes best to think of nature as a
sentient opponent out to get you, and play strategically based on that
assumption. Gould 1963, Beckenkamp 2008 quick overview of criteria
without risk, then say that a better option is to estimate the
probabilities of the different state of nature Agrawal1968 talks about
the strategies in the context of agriculture being a game against nature
here I can slip in the cool thing about acting like the weather is out
to get you being a good idea in states of complete uncertainty
transition -- the problem becomes easier if you have at least some idea
of nature's moves, because then you cna work to get the highest yeilds
given the risk of drought. Decision strategies as choosing different
points -- plot of decisions

\begin{align}
a^2+b^2=c^2
\end{align}

\begin{figure}
\centering
\includegraphics{manuscript_files/figure-latex/unnamed-chunk-2-1.pdf}
\caption{Expected wheat and barley yields under increasing drought risk
with the point of indifference highlighted.}
\end{figure}

If an individual can learn about their environment, such as the risk of
drought in any given year, then they can behave more rationally by
trying to maximize their subjective expected utility. If the probability
distributin of nature's moves is known, the farmer can choose the crop
mix that simply maximizes the expected crop yields given the empirical
frequency distirubtion of nature's moves. In the language of game
theory, this strategy is known as a ``game of ficticious play'' against
nature. This is an effective risk-managing strategy, but like all
similar strategies that adapt to a specific pattern of variability, it
is vulnerable to changes in the pattern of variability (Janssen eta al
2007) This strategy works well when the environment is stationary --
nature plays from a fixed probability distirubtion, but is vulnerable to
environmental nonstationarity. That is, when the mean or higher order
moments of the rainfall distribution shift, playing a game of fictitious
play can backfire because you're too stuck in your ways while the world
changes around you. Thus it is important to understand the dynamics of
risk perception -- how subjective risks, and thus subjective expected
utility -- rise and fall in uncertain environments.

\hypertarget{sec:2}{%
\section{Holocene drought risks in the eastern
Mediterranean}\label{sec:2}}

Estimates of drought risk in the eastern Mediterranean during the the
past 10,000 years were derived from outputs of the TraCE-21k
paleoclimate simulation available on the National Center for Atmospheric
Research. TraCE-21K is a state-of-the-art simulation that uses a coupled
atmosphere-ocean general circulation model (GCM) to recreate the
transient response of the global climate system to changes in the
Earth's orbit and greenhouse gasses over the past 22,000 years from the
Last Glacial Maximum to the present (He 2011). It generates physically
consistent spatiotemporal climate dynamics, driven by current best
estimates of climate forcings (e.g.~orbit, greenhouse gasses, glacial
meltwater flux). The model simulates these dynamics on a six hourly
timescale, and model outputs are archived at a monthly resolution.

TraCE-21k simulation outputs for the past 10,000 years were extracted
from the grid cell overlaying the northern Levant and southeastern
Turkey. This location was selected due to its proximity to
archaeological sites with comparative evidence of crop diversification
strategies. This region includes some of the earliest evidence of crop
domestication, including sites like Jericho and Can Hasan.

Using the climate model output, I divided each model year into dry years
and wet years. A dry year was any year where less than 300mm of rain
fell during the wet season (October-May), the threshold below which
wheat crops will generally fail ({\textbf{???}}), and a wet/normal year
was defined as any year above this threshold. Given the modeled patterns
of wet and dry years, drought risk for any particular year was defined
as the proportion of the previous 50 years that were dry years. Using
the simulated paleoclimate, rather than modern-day weather station
observations or paleoclimate proxies, allows us to estimate not just the
first order statistics of the climate system (mean, variance of dry
years) but also the higher order dynamics like serial autocorrelation in
the perisistence of wet and dry years. This presents a much more
realistic picture of the inherent year-to-year uncertainty in the
climate system, and presents a challenge to simple risk-managing
strategies that assume climatic risks are fixed.

\begin{figure}
\centering
\includegraphics{manuscript_files/figure-latex/unnamed-chunk-5-1.pdf}
\caption{Annual risk of wheat crop failure due to drought, averaged by
fifty-year period. The dashed line indicates the level of risk beyond
which one would plant barley over wheat to maximize subjective expected
yields, given the payoffs defined in Table 1.}
\end{figure}

\hypertarget{sec:3}{%
\section{The dynamics of risk perception}\label{sec:3}}

In order to manage risk, decision-makers must be perceive what exactly
that risk is in the local environment. An individual's perceived risk
entails more than simply the ``objective'' risk in the external
environment (Tucker et al. 2013). A person's perceived risk depends in
large part on their past experience, and how those experiences shape
their expectations for the future. Likewise, the social responses to
risk will refelct the distribution of perceived risks in a population,
and by extension the collective memory of past droughts will have
consequences for the population's survival and evolution.

We can capture the influences of risk perception computationally by
simulating a population of ``belief-based agents.'' These agents develop
beliefs about the risk of drought in their environment through personal
experience. Each agent's perceived drought risks will change their
subjective expected crop yields from planting different ratios of wheat
to barley.

\hypertarget{bayesian-brains}{%
\subsection{Bayesian brains}\label{bayesian-brains}}

We can efficiently simulate these population-level risk perceptions in a
population of agents with Bayesian brains (Kahvalti et al 2019 double
check). Bayesian learning allows for a minimal representation of the
environemnt, so that each agent's beliefs can be encoded in two
parameters -- defining the Beta distribution over their prior beliefs --
rather than the record of all observed years. Not only does this
approach have a clear computational efficiency, but for the same reason
this reflects the processes actually going on in the heads of decision
makers. Our brains do not record every bit of perceived informaiton in
memory, rather it stores a ``compact encoding'' of that information
which it uses for future decisions. Even if individuals aren't --
consciously or unconsciously --- making these calculations -- the basic
algorithmic problem faced by the brain, and solutions it has evolved,
reflect the same basic sets of constraints (Sanborn and Chater 2016).

\begin{figure}
\centering
\includegraphics{manuscript_files/figure-latex/unnamed-chunk-6-1.pdf}
\caption{Development of an individual's perceived drought risk with
time, assuming a fixed drought risk of 0.5. Beliefs are represented as
Beta probabiity distributions, and the increased certainty with age
reflects the varying effective sample size of the Beta prior.}
\end{figure}

\hypertarget{bayesian-learning-from-bernoulli-observations}{%
\subsubsection{Bayesian learning from bernoulli
observations}\label{bayesian-learning-from-bernoulli-observations}}

We can treat this problem elegantly using bayesian statistics using
online bayesian learning from bernoulli objersbations (Bissiri 2010).
Put simply, the probability of a drought occuring in any given year is
treated as a coin flip (i.e.~bernoulli distribution) with paramter theta
reprenting the risk of drought. Uncertainty in the value of theta can be
represented with a beta distribution, constrained between 0 and 1
(Figure XX). We can thus represent the exact information content of each
individual agent's subjective experience of droughts using the diffusion
of the prior belief. What we see here is how an indiviudal's perception
of the risk of drought may start off diffuse at first (here as an
uniformative flat prior in which all values of theta are equally likely.
As each year of experience is observed, the agent updates its beliefs
accordingly, balancing the information in this year's observation with
the cumulative weight of experience.

A simple ``fixed'' bayesian learner thus gets more certain with age. For
young agents with weak priors, the information of each new year can
strongly influence their beliefs. For older agents who have experienced
many more years their priors will be stronger, and they will be less
likely to update their beliefs when balancing the information from a
single year with the many decades of accumulated experience. As was the
case with the simple game of ficitious play above, this strategy is fine
when the environment is stable and risks do not change. But when there
is volatility in the environment, and risks can change, it becomes
important to change your mind. Being too inflexible in one's priors can
lead being too optimistic when things have really changed for the worse,
and alternately being too conservative when things finally do get
better. So from a Bayesian learning standpoint, how should agents choose
to believe new information.

\hypertarget{the-problem-of-nonstationarity}{%
\subsubsection{The problem of
nonstationarity}\label{the-problem-of-nonstationarity}}

That is, what if the risk of a drought, theta, is not a constant but
rather varies unpredictably in time? The statistical problem shifts then
to one of change detection (o'reilly 2013, gallistel et al 2014). In
other words do people simply adapt to changes -- as in the fixed
bayesian learner and moving average approaches -- or do they detect
them? Strict year-to-year updating can result in overly conservative
risk perceptions, suggesting people are slow to change their beliefs
when in reality they often quickly update them (Bissiri 2010). Indeed,
the ``forgetful bayesian brain'' model seems to perform better, with a
dynamic rather than fixed bayesian belief model (Zhang and Yu 2013).

\begin{figure}
\centering
\includegraphics{manuscript_files/figure-latex/unnamed-chunk-7-1.pdf}
\caption{Change in perceived drought risks in an older (50) and younger
(15) agent before and after a 25-year dry period.}
\end{figure}

So what are the consequences of these cognitive processes for early
farmers in the Mediterranean.

Simulate a time series with varying drought frequency. This is a
stepwise nonstationary process. Setup an experiment exploring 6
scenarios.

This suggests that people are more in danger of not realizing conditions
are improving when they actually have -- the opportunity cost -- rather
than failing to realize the change in the firstplace. The law of small
numbers is very important here. The range of each person's experience is
so small -- every individual's perceived risk will be a small and noisy
sample of the true value. Older people are much more certain, even
though they are wrong.

\includegraphics{manuscript_files/figure-latex/unnamed-chunk-12-1.pdf}

These two plots begin to look more alike the more you sample, which
makes sense because its an ergodic system.

Note in these plots above and below we're summing all the wheat and
barley in the population, compared to below which shows the wheat
proportions of individuals.

\hypertarget{discussion-and-conclusions}{%
\section{Discussion and Conclusions}\label{discussion-and-conclusions}}

The bias-corrected wet season precipitation outputs from TraCE-21k show
a relatively stationary distribution over the past 10,000 years, with
slightly drier conditions during the early-mid Holocene transition at
c.a. 8ka BP (Figure 5). The utility of a diversified wheat-barley crop
is immediately apparent; rainfall often fails to meet the 300mm
threshold for a productive wheat crop during the entire 10,000 year
span, but drops below the threshold for barley only a handful of years.

The risk of crop failure due to drought in any given year varies between
10\% and 20\%, punctuated by one to four century-long events where risk
drops as low as 6\% and or exceeds 25\% (Figure 6). Centuries of high
drought risk cluster around known periods of regional and global climate
deterioration, such as at 8.6ka BP and 4.2ka BP.

Given the average drought risks estimated from TraCE-21k and assuming
perfect monitoring and recall of recent droughts (i.e. \(\lambda = 0\)),
farmers intent on minimizing the risk of crop failure would be expected
to plant about 65\% barley to 35\% wheat on average. This proportion is
within the range estimated from macrobotanical remains at a nearby
archaeological site ({\textbf{???}}). Yield maximizing farmers instead
play a pure strategy profile, planting wheat if the perceived drought
risk is less than 0.68 and barley if it is more. Because the
GCM-simulated drought risk never rises above this threshold, a
yield-maximizing farmer will always plant a wheat monocrop.

Examining the long-term dynamics of crop yields among all
decision-making strategies and psychological profiles reveals
variability in yields due to different decision making preferences is
far greater variability due to differences in risk perception (Figure
7). Risk-minimizing strategies sacrifice productivity for
predictability, consistently lowering the mean and variance of crop
yields over time. Risk-minimizing strategies are more sensitive to risk
perception than yield-maximizing strategies; varying the memory decay
rate parameter \(\lambda\) had no effect on the optimal crop allocation
for yield maximizers.

During periods of climatic stability, allowing past experiences to
influence decision making helps farmers minimize the impacts of
\emph{predictable} drought. But past experiences are less informative
during periods of rapid climate change, and even farmers who manage risk
``optimally'' experience major food shortfalls. Climatically-induced
variability in food supplies consistently surpasses that from
differences in risk-perception psychology, but not that from different
risk aversion preferences. This finding suggests that the efficacy of
risk managing strategies that rely on regulatory feedbacks is limited to
periods of relative climatic stability. This is consistent with the
observation from the Roman Period Mediterranean that while minor food
crises were common in the ancient world, extreme famines were rare
(Garnsey 1989).

Statistically downscaling palaeoclimate simulations and coupling them to
game theoretic models of decision making under uncertainty is a simple
way to better understand the patterning of climate changes and the
regulatory feedbacks farmers use to adapt to them. To further explore
these patterns, future work should incorporate additional
risk-minimizing strategies such as storage and exchange. Food storage
and exchange can also be thought of as forms of diversification, the
former in the time and the latter in space ({\textbf{???}}). Food
exchange would be of particular interest from a game theoretic
perspective, as incomplete information with respect to an exchange
partner's crop diversification and storage practices can lead to
situations of moral hazard.

Think about the importance of the population's age structure, becuase
that's going to impact the collective memory of drought.

\setlength{\parindent}{-0.5in}
\setlength{\leftskip}{0.5in}
\setlength{\parskip}{8pt}

\hypertarget{references}{%
\section*{References}\label{references}}
\addcontentsline{toc}{section}{References}

\hypertarget{refs}{}
\leavevmode\hypertarget{ref-Garnsey1989}{}%
Garnsey, Peter. 1989. \emph{Famine and Food Supply in the Graeco-Roman
World: Responses to Risk and Crisis}. Cambridge University Press.

\leavevmode\hypertarget{ref-GOULD1963a}{}%
Gould, Peter R. 1963. ``Man Against His Environment: A Game Theoretic
Framework.'' \emph{Annals of the Association of American Geographers} 53
(3): 290--97. \url{https://doi.org/10.1111/j.1467-8306.1963.tb00450.x}.

\leavevmode\hypertarget{ref-Halstead1989}{}%
Halstead, Paul, and John O'Shea. 1989. ``Introduction: cultural
responses to risk and uncertainty.'' In \emph{Bad Year Economics:
Cultural Responses to Risk and Uncertainty}, edited by Paul Halstead and
John O'Shea, 1--7. Cambridge: Cambridge University Press.
\url{https://doi.org/10.2307/281441}.

\leavevmode\hypertarget{ref-He2011}{}%
He, Feng. 2011. ``Simulating Transient Climate Evolution of the Last
Deglatiation with CCSM3.'' PhD thesis, Madison.

\leavevmode\hypertarget{ref-Tucker2013}{}%
Tucker, Bram, Jaovola Tombo, Tsiazonera, Patricia Hajasoa, Charlotte
Nagnisaha, Vorisoa Rene Lahitoka, and Christian Zahatsy. 2013. ``Beyond
Mean and Variance: Economic Risk Versus Perceived Risk of Farming,
Foraging, and Fishing Activities in Southwestern Madagascar.''
\emph{Human Ecology} 41 (3). Springer US: 393--407.
\url{https://doi.org/10.1007/s10745-013-9563-2}.

\bibliographystyle{spbasic}
\bibliography{bibliography.bib}

\end{document}
