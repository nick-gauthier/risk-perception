% !TeX program = pdfLaTeX
\documentclass[smallextended]{svjour3}       % onecolumn (second format)
%\documentclass[twocolumn]{svjour3}          % twocolumn
%
\smartqed  % flush right qed marks, e.g. at end of proof
%
\usepackage{amsmath}
\usepackage{graphicx}
\usepackage[utf8]{inputenc}

\usepackage[hyphens]{url} % not crucial - just used below for the URL
\usepackage{hyperref}
\providecommand{\tightlist}{%
  \setlength{\itemsep}{0pt}\setlength{\parskip}{0pt}}

%
% \usepackage{mathptmx}      % use Times fonts if available on your TeX system
%
% insert here the call for the packages your document requires
%\usepackage{latexsym}
% etc.
%
% please place your own definitions here and don't use \def but
% \newcommand{}{}
%
% Insert the name of "your journal" with
% \journalname{myjournal}
%

%% load any required packages here





\begin{document}

\title{The dynamics of risk perception in a Mediterranean agroecosystem \thanks{Grants or other notes about the article that should go on the front page
should be placed here. General acknowledgments should be placed at the
end of the article.} }


    \titlerunning{Risk perception in a Mediterranean agroecosystem}

\author{  Nicolas Gauthier 1 \and  }

    \authorrunning{ Gauthier }

\institute{
        Nicolas Gauthier 1 \at
     Laboratory of Tree-Ring Research \& School of Geography and Development,
 University of Arizona \\
     \email{\href{mailto:ngauthier@email.arizona.edu}{\nolinkurl{ngauthier@email.arizona.edu}}}  %  \\
%             \emph{Present address:} of F. Author  %  if needed
    \and
    }

\date{Received: date / Accepted: date}
% The correct dates will be entered by the editor


\maketitle

\begin{abstract}
Small-scale agriculturalists in the Mediterranean Basin rely on multiple
strategies including diversification, intensification, and storage to
maintain a stable food supply in the face of environmental uncertainty.
Each of these strategies requires farmers to make specific resource
allocation decisions in response to environmental risks and is thus
sensitive to variability in both the spatiotemporal pattern of risk and
the ability of farmers to perceive that pattern. In this chapter, I
present an agent-based model of a Mediterranean agroecosystem. By
driving the model with realistic environmental dynamics derived from
simulations of mid-Holocene Mediterranean climate, and by allowing the
psychology of risk perception to vary among individual farmers, I
explore the hidden vulnerabilities of traditional risk-management
strategies to periods of rapid climate change. I show that even when
farmers are able to manage risk ``optimally'' in light of past
experience, unanticipated changes in the spatiotemporal pattern of
rainfall can still lead to major food shortfalls.
\\
\keywords{
        key \and
        dictionary \and
        word \and
    }

    \subclass{
                    MSC code 1 \and
                    MSC code 2 \and
            }

\end{abstract}


\def\spacingset#1{\renewcommand{\baselinestretch}%
{#1}\small\normalsize} \spacingset{1}


\hypertarget{intro}{%
\section{Introduction}\label{intro}}

Over the past 10,000 years, subsistence farmers in the Mediterranean
basin have developed sophisticated strategies to maintain stable food
supplies given uncertain rainfall. Their suite of strategies include
practices like crop diversification, storage, and exchange (Halstead and
O'Shea 1989)). These risk-management strategies all represent forms of
\emph{regulatory feedback} used to manage complex socio-ecological
systems. Regulatory feedbacks convert a system's high-variance inputs
into low-variance outputs by dynamically monitoring the outputs and
adjusting internal properties of the system ({\textbf{???}}).
Risk-managing strategies require farmers to make specific resource
allocation decisions in response to environmental risks, and as
regulatory feedbacks they are thus sensitive to variability in both the
spatiotemporal patterns of risk and the ability of farmers to perceive
and act on those patterns.

Crop diversification is an excellent example of a widespread and
effective regulatory feedback strategy (Figure 1). Relying on a mix of
food types with different climatic tolerances is an efficient way to
maintain a robust food supply ({\textbf{???}},Anderies2006). In the
Mediterranean, land-use strategies involving a diversified portfolio of
wheat and barley have been employed by even the earliest sedentary
farmers, and continue to be used to this day (Gould
1963,Slafer1999,Abbo2009a,Marston2011190). Wheat generally has higher
yields but is sensitive to water deficits, while barley has lower yields
and is drought tolerant. Planting a mix of wheat and barley, either in
the same plot or in a combination of plots, is an effective means of
diversifying grain supplies. By adjusting the ratio of wheat to barley,
farmers can adapt to changing drought risks.

The secret to the success of earliest farmers on the mediterrenan
littoral was in their manipulation of the local ecosystem to manage
environmental risks. Early farmers constructed a complex agroecosystem
to manage risk. They needed to maintain stable crop yields in face of
uncertain drought risks. Yield stability is very important for
traditional agriculturlal commmunities not connected to larger bulk food
networks (Abbo et al 2010) neolithic crop package -- weiss zohary 2011
cite abbo et al 2010? for more stuff about ladraces Wheat/barley systems
example of system that combines low yield, low wirks crop with high
yield high risk crop, an effective context paut et al 2019 These
strategies work when risks are knonw and constant, but may be more
susceptible to periods of changing climateic risks so understanding how
people of the past adapted to change is important to us today

Are these crop-diversification strategies vulnerable to the same
dynamics as other social-ecological systems with similar regulatory
feedbacks mechanisms? One way to address this is to model the influence
of imperfect monitoring and biased decision making in uncertain
environments. In this study, I accomplish this by answering two main
questions:

\begin{enumerate}
\def\labelenumi{\arabic{enumi}.}
\tightlist
\item
  What was the temporal pattern of climate and climate change during the
  Holocene in the eastern Mediterranean?
\item
  How well could crop diversification strategies cope with this pattern,
  given farmers' imperfect risk perception and decision-making?
\end{enumerate}

Questions: How would the earliest farmers have made decisions (dealt
with uncertainty?) about crop diversification? ie what were the
different algorithms? -- better understand the decision context What are
the consequences of theses different decisions (i.e.~how well did they
perform) at different points in the history of the mediterranean How
were farmers able to detect changes in the environment, rather than
unconsciously adapt to them? what are the different consequences we'd
expect for each THIS is the most important question because with it I
can frame the results around and individual's perception of droughts in
the neolithic, and the populaiton-level distribution of drought
perceptions at given times

I couple a climate model with high temporal resolution to a simple game
theoretic model of agricultural decision making under uncertainty, in
order to examine the performance of optimal and suboptimal wheat-barley
diversification practices. I model the year-to-year crop diversification
strategies made by farmers as an iterated game of fictitious play
against nature (Gould 1963). Farmers are boundedly rational, in that
they seek to maximize their objective functions given the perceived
probabilities of different actions by nature, but work with flawed
mental models of climate-related risks due to imperfect recall of past
events. Furthermore, by using a climate model to represent the actions
of nature, I can more precisely capture the characteristic
autocorrelated patterns of rainfall variability, rather than simply
drawing rainfall values from a static distribution. This framework
allows me to address the sensitivity of diversification strategies to
changes and climatic variability and imperfect human perceptions of
those changes.

\hypertarget{sec:1}{%
\section{Decision-making in a game against nature}\label{sec:1}}

The basic decision-making problem facing a farmer seeking to diversify
their crops can be thought of as a game against nature. We can represent
this simplified decision context as a ``game'' in a game theorietic
context. cite luce raiffa 1989 for game against nature, and milnor 1952,
Agrawal1968, cassidy1971, gould 1963 The ``game'' in this context is the
farmer's decision of which crops to plant and in what proportions, given
uncertianty in the future ``state-of-nature'', and nature varies between
several possible states such as dry and wet years. payoff matrix goes
here in reality continuous, but cite behavioral econ for people often
intuitively solve an easier problem when faced with a complex real-world
situation The exact values in the payoff matrix here are less important
than the relative payoffs in each quadrant. We can think of this game as
being the culturally-inherited object. our question is how do
individuals make decisions here? so in this case the cultural norm would
be to mix wheat and barley, but the learning is in the exact ratio --
both individual and social learning influence perceptions of eindividual
vs social learning tucker 2007

Estimates of yield volume (t/ha) for each crop type in wet and dry years
were derived from isotopic studies of ancient wheat and barley samples
from archaeological sites in the Mediterranean (Araus, Slafer, and
Romagosa 1999) (Table 1).

\begin{table}
\centering
\caption{Estimated yields for ancient wheat and barley varieties derived from [@Slafer1999].}
\begin{tabular}{|l|l|l|}
\hline

 & Dry Year & Normal Year \\ \hline
Barley Yield & 0.93 & 1.18 \\ \hline
Wheat Yield & 0 & 1.60 \\ \hline

\end{tabular}
\end{table}

How should an early farmer make a decision in light ot this uncertainty?
The basic logic of crop diversitication is Portfolio of crops ala modern
portfolio theory: Blank2001 Sometimes best to think of nature as a
sentient opponent out to get you, and play strategically based on that
assumption. Gould 1963, Beckenkamp 2008 quick overview of criteria
without risk, then say that a better option is to estimate the
probabilities of the different state of nature Agrawal1968 talks about
the strategies in the context of agriculture being a game against nature
here I can slip in the cool thing about acting like the weather is out
to get you being a good idea in states of complete uncertainty
transition -- the problem becomes easier if you have at least someI idea
of nature's moves , because then you cna work to get the highest yeilds
given the risk of drought Decision strategies as choosing different
points -- plot of decisions

\begin{align}
a^2+b^2=c^2
\end{align}

\begin{figure}
\centering
\includegraphics{manuscript_files/figure-latex/unnamed-chunk-4-1.pdf}
\caption{Expected wheat and barley yields under increasing drought risk
with the point of indifference highlighted.}
\end{figure}

If an individual can learn about their environment, such as the risk of
drought in any given year, then they can behave more rationally by
trying to maximize their subjective expected utility If the probability
distributin of nature's moves is known, the farmer can choose the crop
mix that simply maximizes the expected crop yields given the empirical
frequency distirubtion of nature's moves. In the language of game
theory, this strategy is knoan as a ``game of ficticious play'' against
nature. This is an effective risk-managing strategy, but like all
similar strategies that adapt to a specific pattern of variability, it
is vulnerable to changes in the pattern of variability (Janssen eta al
2007) This strategy works well when the environment is stationary --
nature plays from a fixed probability distirubtion, but is vulnerable to
environmental nonstationarity. That is, when the mean or higher order
moments of the rainfall distribution shift, playing a game of fictitious
play can backfire because you're too stuck in your ways while the world
changes around you. Thus it is important to understand the dynamics of
risk perception -- how subjective risks, and thus subjective expected
utility -- rise and fall in uncertain environments.

\hypertarget{sec:2}{%
\section{The dynamics of risk perception}\label{sec:2}}

\includegraphics{manuscript_files/figure-latex/unnamed-chunk-5-1.pdf}
\includegraphics{manuscript_files/figure-latex/unnamed-chunk-5-2.pdf}

\hypertarget{drought-risks-in-the-middle-holocene}{%
\section{Drought risks in the middle
Holocene}\label{drought-risks-in-the-middle-holocene}}

\hypertarget{climate-risks-and-risk-perception}{%
\section{Climate Risks and Risk
Perception}\label{climate-risks-and-risk-perception}}

Using the bias-corrected climate model output, I divided each model year
into dry years and wet years. A dry year was any year where less than
300mm of rain fell during the wet season (October-March), the threshold
below which wheat crops will generally fail ({\textbf{???}}), and a
wet/normal year was defined as any year above this threshold.

Given the modeled patterns of wet and dry years, drought risk for any
particular year was defined as the proportion of the previous 50 years
that were dry years

\begin{equation}
    P_{dry} = \frac{\sum_{n=t-1}^{t-50} precip_n < 300}{50},
\end{equation} where \(t\) is the current time step. The 50 year time
span was selected to approximate the accumulated observational knowledge
of an individual farmer and their immediate household. As a result, this
approach does not allow for accumulated social learning, although it
could easily be extended to do so in future studies.

\hypertarget{climate-modeling-and-bias-correction}{%
\subsection{Climate Modeling and Bias
Correction}\label{climate-modeling-and-bias-correction}}

Estimates of drought risk in the eastern Mediterranean during the the
past 10,000 years were derived from outputs of the TraCE-21k
paleoclimate simulation available on the National Center for Atmospheric
Research's Earth System Grid repository
(\url{https://www.earthsystemgrid.org}). TraCE-21K is a state-of-the-art
simulation that uses a coupled atmosphere-ocean general circulation
model (GCM) to recreate the transient response of the global climate
system to orbital parameters and greenhouse gasses over the past 22,000
years from the Last Glacial Maximum to the present (Figure 2) (He 2011).
It generates physically consistent spatiotemporal climate dynamics,
driven by current best estimates of climate forcings (e.g.~orbit,
greenhouse gasses, glacial meltwater flux). The model simulates these
dynamics on a six hourly timescale, and model outputs are archived at a
monthly resolution.

TraCE-21k simulation outputs for the past 10,000 years were
bias-corrected using the CDF-t method and observed precipitation from a
weather station in near the town of Salihli in western Turkey. This
location was selected due to its long (\textasciitilde{}70yr)
observational record, the representativeness of western Anatolian
climate for the greater eastern Mediterranean, and its proximity to
archaeological sites with comparative evidence of crop diversification
strategies.

Now incorporate realistic drought variability from simulations.

\begin{verbatim}
## Loading required namespace: ncdf4
\end{verbatim}

\begin{figure}
\centering
\includegraphics{manuscript_files/figure-latex/unnamed-chunk-8-1.pdf}
\caption{Annual risk of wheat crop failure due to drought, averaged by
fifty-year period}
\end{figure}

\hypertarget{the-model}{%
\section{The Model}\label{the-model}}

Run an experiment. Simulate a time series with varying drought
frequency. This is a stepwise nonstationary process

\includegraphics{manuscript_files/figure-latex/unnamed-chunk-12-1.pdf}

Redo these plots so its a single person, one who was born in a drought
period then experiences an ameliorarion and another who experiences a
shift to drought conditions later in life? basically this section neds
to work more with individuals and explore that stuff

\includegraphics{manuscript_files/figure-latex/unnamed-chunk-14-1.pdf}

\includegraphics{manuscript_files/figure-latex/unnamed-chunk-15-1.pdf}

\includegraphics{manuscript_files/figure-latex/unnamed-chunk-16-1.pdf}

\includegraphics{manuscript_files/figure-latex/unnamed-chunk-17-1.pdf}

\includegraphics{manuscript_files/figure-latex/unnamed-chunk-18-1.pdf}

These two plots begin to look more alike the more you sample, which
makes sense because its an ergodic system.

Note in these plots above and below we're summing all the wheat and
barley in the population, compared to below which shows the wheat
proportions of individuals
\includegraphics{manuscript_files/figure-latex/unnamed-chunk-19-1.pdf}
\includegraphics{manuscript_files/figure-latex/unnamed-chunk-19-2.pdf}

\includegraphics{manuscript_files/figure-latex/unnamed-chunk-20-1.pdf}
\includegraphics{manuscript_files/figure-latex/unnamed-chunk-20-2.pdf}

\includegraphics{manuscript_files/figure-latex/unnamed-chunk-21-1.pdf}
\includegraphics{manuscript_files/figure-latex/unnamed-chunk-21-2.pdf}

\begin{verbatim}
## Picking joint bandwidth of 0.0146
\end{verbatim}

\includegraphics{manuscript_files/figure-latex/unnamed-chunk-22-1.pdf}

\begin{verbatim}
## Picking joint bandwidth of 0.0204
\end{verbatim}

\includegraphics{manuscript_files/figure-latex/unnamed-chunk-22-2.pdf}

make this a function to calculate payoffs for all wheat or all barley
strategies as a comparison
\includegraphics{manuscript_files/figure-latex/unnamed-chunk-23-1.pdf}

\includegraphics{manuscript_files/figure-latex/unnamed-chunk-24-1.pdf}

\section{Results and Discussion}

The bias-corrected wet season precipitation outputs from TraCE-21k show
a relatively stationary distribution over the past 10,000 years, with
slightly drier conditions during the early-mid Holocene transition at
c.a. 8ka BP (Figure 5). The utility of a diversified wheat-barley crop
is immediately apparent; rainfall often fails to meet the 300mm
threshold for a productive wheat crop during the entire 10,000 year
span, but drops below the threshold for barley only a handful of years.

The risk of crop failure due to drought in any given year varies between
10\% and 20\%, punctuated by one to four century-long events where risk
drops as low as 6\% and or exceeds 25\% (Figure 6). Centuries of high
drought risk cluster around known periods of regional and global climate
deterioration, such as at 8.6ka BP and 4.2ka BP.

Given the average drought risks estimated from TraCE-21k and assuming
perfect monitoring and recall of recent droughts (i.e. \(\lambda = 0\)),
farmers intent on minimizing the risk of crop failure would be expected
to plant about 65\% barley to 35\% wheat on average. This proportion is
within the range estimated from macrobotanical remains at a nearby
archaeological site ({\textbf{???}}). Yield maximizing farmers instead
play a pure strategy profile, planting wheat if the perceived drought
risk is less than 0.68 and barley if it is more. Because the
GCM-simulated drought risk never rises above this threshold, a
yield-maximizing farmer will always plant a wheat monocrop.

Examining the long-term dynamics of crop yields among all
decision-making strategies and psychological profiles reveals
variability in yields due to different decision making preferences is
far greater variability due to differences in risk perception (Figure
7). Risk-minimizing strategies sacrifice productivity for
predictability, consistently lowering the mean and variance of crop
yields over time. Risk-minimizing strategies are more sensitive to risk
perception than yield-maximizing strategies; varying the memory decay
rate parameter \(\lambda\) had no effect on the optimal crop allocation
for yield maximizers.

During periods of climatic stability, allowing past experiences to
influence decision making helps farmers minimize the impacts of
\emph{predictable} drought. But past experiences are less informative
during periods of rapid climate change, and even farmers who manage risk
``optimally'' experience major food shortfalls. Climatically-induced
variability in food supplies consistently surpasses that from
differences in risk-perception psychology, but not that from different
risk aversion preferences. This finding suggests that the efficacy of
risk managing strategies that rely on regulatory feedbacks is limited to
periods of relative climatic stability. This is consistent with the
observation from the Roman Period Mediterranean that while minor food
crises were common in the ancient world, extreme famines were rare
(Garnsey 1989).

Statistically downscaling palaeoclimate simulations and coupling them to
game theoretic models of decision making under uncertainty is a simple
way to better understand the patterning of climate changes and the
regulatory feedbacks farmers use to adapt to them. To further explore
these patterns, future work should incorporate additional
risk-minimizing strategies such as storage and exchange. Food storage
and exchange can also be thought of as forms of diversification, the
former in the time and the latter in space ({\textbf{???}}). Food
exchange would be of particular interest from a game theoretic
perspective, as incomplete information with respect to an exchange
partner's crop diversification and storage practices can lead to
situations of moral hazard.

Think about the importance of the population's age structure, becuase
that's going to impact the collective memory of drought.

\hypertarget{references}{%
\section*{References}\label{references}}
\addcontentsline{toc}{section}{References}

\hypertarget{refs}{}
\leavevmode\hypertarget{ref-Slafer1999}{}%
Araus, J. L., G. A. Slafer, and I. Romagosa. 1999. ``Durum wheat and
barley yields in antiquity estimated from 13C discrimination of
archaeological grains: a case study from the Western Mediterranean
Basin.'' \emph{Australian Journal of Plant Physiology} 26 (4): 345.
\url{https://doi.org/10.1071/PP98141}.

\leavevmode\hypertarget{ref-Garnsey1989}{}%
Garnsey, Peter. 1989. \emph{Famine and Food Supply in the Graeco-Roman
World: Responses to Risk and Crisis}. Cambridge University Press.

\leavevmode\hypertarget{ref-GOULD1963a}{}%
Gould, Peter R. 1963. ``Man Against His Environment: A Game Theoretic
Framework.'' \emph{Annals of the Association of American Geographers} 53
(3): 290--97. \url{https://doi.org/10.1111/j.1467-8306.1963.tb00450.x}.

\leavevmode\hypertarget{ref-Halstead1989}{}%
Halstead, Paul, and John O'Shea. 1989. ``Introduction: cultural
responses to risk and uncertainty.'' In \emph{Bad Year Economics:
Cultural Responses to Risk and Uncertainty}, edited by Paul Halstead and
John O'Shea, 1--7. Cambridge: Cambridge University Press.
\url{https://doi.org/10.2307/281441}.

\leavevmode\hypertarget{ref-He2011}{}%
He, Feng. 2011. ``Simulating Transient Climate Evolution of the Last
Deglatiation with CCSM3.'' PhD thesis, Madison.

\bibliographystyle{spbasic}
\bibliography{bibliography.bib}

\end{document}
